%% DO NOT CHANGE
\documentclass[twoside,9pt,journal,letterpage]{IEEEtran}
\usepackage{cite}
\usepackage{amsmath,amssymb,amsfonts}
\usepackage{algorithmic}
\usepackage{graphicx}
\usepackage{textcomp}
\def\BibTeX{{\rm B\kern-.05em{\sc i\kern-.025em b}\kern-.08em
    T\kern-.1667em\lower.7ex\hbox{E}\kern-.125emX}}
%% DO NOT CHANGE

% Run through MikTeX TexWorks once with auto-install missing packages enabled.

\let\labelindent\relax
\usepackage[shortlabels]{enumitem}
\usepackage{float}
\setlength{\textfloatsep}{5pt}

\usepackage{datetime}
\newdateformat{monthdayyeardate}{%
  \monthname[\THEMONTH]~\THEDAY, \THEYEAR}

\newcommand{\titlestr}{Adaptive Clocking Techniques for SoC Supply Droop Response in Predictive 7nm CMOS}
\title{\titlestr}

\usepackage{lipsum}

%\author{
%\authorblockN{Dan Fritchman}
%\authorblockA{Department of Electrical Engineering and Computer Sciences\\
%University of California, Berkeley\\
%Berkeley, California  94720\\
%Email: dan\_fritchman@berkeley.edu}
%\and
%\authorblockN{Wahid Rahman}
%\authorblockA{Department of Electrical Engineering and Computer Sciences\\
%University of California, Berkeley\\
%Berkeley, California  94720\\
%Email: wahid.rahman@berkeley.edu}
%}

\author{
	Dan Fritchman, \IEEEmembership{Member, IEEE} and Wahid Rahman, \IEEEmembership{Member, IEEE}
	\thanks{Date of publication \monthdayyeardate\today.}
	\thanks{
		D. Fritchman and W. Rahman are with the Department of Electrical Engineering and Computer Sciences, University of California, Berkeley, Berkeley, CA 94720 USA (e-mail: dan\_fritchman@berkeley.edu; wahid.rahman@berkeley.edu).}
	\thanks{
		This work was supported by Professor Borivoje Nikoli\'{c}.}
}

\markboth{UC Berkeley Proceedings of EE241B, March 2020}{Fritchman \MakeLowercase{\textit{and}} Rahman: \titlestr}

\pubid{~\copyright~2020 University of California, Berkeley}

\begin{document}
\maketitle
\IEEEpeerreviewmaketitle

\begin{abstract}
\lipsum[1]
\end{abstract}

\begin{IEEEkeywords}
Adaptive clocking, adaptive frequency, power efficiency, supply-droop mitigation, supply-voltage droop.
\end{IEEEkeywords}

\maketitle

\section{Introduction}

Power management techniques in modern system-on-chips (SoCs) are critical for energy-efficient processors ranging from data servers to mobile devices. SoC thermal dissipation constraints and energy-saving modes necessitate system-level power management to decrease the power supply or reduce the number of active processing cores. Such techniques exhibit a decrease (i.e. droop) in the SoC supply voltage ($V_{DD}$) due to: the controlled decrease of $V_{DD}$ from a supply regulator or DC-DC converter; and the transient $L \frac{di}{dt}$ supply voltage ripple due to sudden current changes through package inductances when dynamically enabling or disabling on-chip processing cores. To maximize processing throughput, SoCs are designed to operate close to the maximum possible clock frequency ($f_{MAX}$) for the target ($V_{DD}$) with minimal guardbanding. Decreasing $V_{DD}$ reduces the drain currents of CMOS transistors, thereby increasing propagation delays in the critical paths of digital logic. If sufficient timing margin is not available in the design, these critical paths can fail to meet timing and cause unrecoverable errors.

\begin{figure}[h]
	\centering
	\includegraphics[width=\columnwidth]{fig_fmax}
	\caption{Reduction of $V_{DD}$ and corresponding reduction in $f_{MAX}$ (adapted from \cite{ahmad2017}).}
	\label{fig:fmax}
\end{figure}
\begin{figure}[h]
	\centering
	\includegraphics[width=\columnwidth]{fig_droop_schem}
	\caption{Supply droop transients due to package inductances and load current (adapted from \cite{hashimoto2018}).}
	\label{fig:droop1}
\end{figure}
\begin{figure}[h]
	\centering
	\includegraphics[width=\columnwidth]{fig_droop}
	\caption{Supply droop transients due to package inductances and load current (adapted from \cite{hashimoto2018}).}
	\label{fig:droop2}
\end{figure}

To mitigate such failures, adaptive clock generation techniques have emerged in recent generations of high-performance SoCs\cite{ahmad2017,hashimoto2018,wilcox2015,floyd2017,bowman2016}. Adaptive-clock systems detect transient supply events and dynamically adjust clock frequencies to not exceed a changing $f_{MAX}$ as $V_{DD}$ decreases. As Fig.\ \ref{fig:fmax} illustrates, a decrease from high to low $V_{DD}$ to save power corresponds to a decrease in $f_{MAX}$, and the SoC clock frequency ($f_{CLK}$) must track this change without exceeding the correct $f_{MAX}$ (overclocking) and with minimal guardbanding (underclocking) to minimize processing throughput loss during this change\cite{ahmad2017}. 

Transient response of adaptive clock techniques is also vital. The external package routes supplying power to the SoC impact supply voltage settling during changes to on-chip core utilization \cite{hashimoto2018}, as illustrated in Fig.\ \ref{fig:droop1}. The range of inductances \& capacitances in this network induces a medley of fast and slow time constants. Resilient adaptive clock generation circuits react quickly to fast ripples in $V_{DD}$ ranging from 50-100MHz \cite{hashimoto2018,wilcox2015} while also tracking slower transient settling at 1MHz and below \cite{hashimoto2018,bowman2016,wilcox2015}.

% If using pudid, place pubidadjcol so it lands in second column of first page
\pubidadjcol

Several schemes exist to realize supply-induced clock adaptation ranging from directly controlling the clock-synthesizing phase-locked loop (PLL)\cite{ahmad2017,hashimoto2018} to modulating a tunable delay line or phase rotator downstream in an adaptive clock distribution (ACD) network\cite{bowman2016,floyd2017,wilcox2015}. In this work, we survey and contrast the adaptive PLL- and ACD-based mechanisms presented in \cite{hashimoto2018} and \cite{wilcox2015}, respectively. This brief is organized as follows: Section \ref{sec:overview} provides an overview of adaptive clocking schemes and defines the scope of this work; Section \ref{sec:details} discusses the operating principles of mechanisms presented in \cite{hashimoto2018} and \cite{wilcox2015}; Section \ref{sec:comparison} provides a comparison of \cite{hashimoto2018} and \cite{wilcox2015}; and finally Section \ref{sec:conclusion} discusses planned work and concludes this brief.

\section{Adaptive Clocking Schemes}
\label{sec:overview}

Adaptive clocking systems include two fundamental components:

\begin{enumerate}[(a)]
\item a \textit{power-supply sensor}, which measures and
reports transient droops in supply voltage; and
\item a \textit{clock period actuator}, which modulates the
system clock period in response to reports from
the power-supply sensor.
\end{enumerate}

During a supply drop, low-latency sensing and actuation is crucial to detect and adjust the clock frequency quickly to ensure error-free operation continues throughout the event. If and when $V_{DD}$ recovers from this event to its nominal value, additional circuitry can assist with managing the corresponding $f_{CLK}$ recovery.

ACD- and adaptive PLL-based systems differ in their implementation of the clock period actuator. ACD-based systems decrease the clock frequency by extending clock periods using an ACD circuit placed between the clock-synthesizing PLL and the global clock distribution network as illustrated in Fig.\ \ref{fig:overview_acd}. A digital code $CTRL_{FCLK}$ from the power supply sensor controls the ACD circuit such that it extends the period of the synthesized clock, $CLK_{PLL}$, to the desired frequency given the sensed change in $V_{DD}$. This lower-frequency clock, $CLK_{SOC}$, is then distributed through the global clock network to the SoC processing cores. Examples of this ACD circuit include tunable-length delay \cite{bowman2016} and phase rotators \cite{wilcox2015}, of which the latter is discussed in Section \ref{sec:details_acd}.

\begin{figure}[h]
	\centering
	\includegraphics[width=0.7\columnwidth]{fig_overview_acd}
	\caption{Adaptive clock distribution (ACD) based system.}
	\label{fig:overview_acd}
\end{figure}

\begin{figure}[h]
	\centering
	\includegraphics[width=\columnwidth]{fig_overview_pll}
	\caption{Adaptive PLL-based system with possible intra-PLL control variants.}
	\label{fig:overview_pll}
\end{figure}

Adaptive PLL-based systems, in contrast, incorporate information from the power-supply sensor to dynamically change the behaviour of PLL sub-components. Typically, these schemes affect oscillator control on top of the traditional PLL loop. In \cite{ahmad2017} for example, the oscillator is controlled indirectly through a loop filter modified for droop response: when a droop is detected, the traditional proportional- and integral-loop filter switches to proportional-only to form a type-I PLL that tunes the oscillator to frequency lock without hazardous over- or underclocking. In \cite{hashimoto2018}, this idea is further extended by interrupting the PLL feedback loop to modify the oscillator directly, reacting to a supply droop event with relatively low latency. The feedback divider is then slowly modified to restore PLL lock. This latter work is further discussed in Section \ref{sec:details_pll}.

The focus of this brief is to compare the design and effectiveness of clock period actuators in ACD and PLL-based schemes. While high-resolution power-supply sensors and their low-latency integration are critical to the overall adaptation time of such schemes, it remains beyond the scope of this work.

\section{Operating principles}
\label{sec:details}
The two main works for comparison are the PLL- and ACD-based adaptive clocking schemes in \cite{hashimoto2018} and \cite{wilcox2015} respectively. The operating principles of each are described in this section.

\subsection{PLL-Based Adaptive Clocking}
\label{sec:details_pll}
While adaptive clocking has been supported by analog PLLs such as \cite{kurd2009next}, its cost of implementation is far lower for digital PLLs. The scheme reported in \cite{hashimoto2018} presents a PLL-based adaptive clocking control for three SPARC processor cores in a 20nm CMOS testchip. A sense point placed near one of the SPARC cores transmits the core $V_{DD}$ through low-impedance on-package (off-chip) routing to the on-chip power-supply sensor. This sensor converts $V_{DD}$ droops into a thermometer-coded quantized digital signal, $q[n]$, by use two calibrated delay lines and a 8-bit time-to-digital converter (TDC). 

A major contribution of \cite{hashimoto2018} is to use the instantaneous value of q[n] to immediately reduce the PLL's oscillator frequency to correct for high-frequency droops and use a filtered version of $q[n]$ to correct for low frequency droops and ultimately re-lock the PLL feedback loop. As shown in Fig.\ \ref{fig:detail_pll} and Fig.\ \ref{fig:detail_pllarch}, $q[n]$ is processed by the frequency control logic to generate two control signals: $\Delta F_{code}$ that can instantaneously respond to changes in $q[n]$ to directly change the digitally-controlled oscillator (DCO) frequency, and $N^{sync}_{code}$ that relies only on a filtered value of $q[n]$ to track low-frequency droops and re-lock the PLL loop at the new frequency by changing the feedback division value. The minimum function and the non-linear filter together effectively construct an all-pass filter when $V_{DD}$ decreases and a low-pass filter when $V_{DD}$ increases. They allow the loop to actuate the clock with minimal latency during a $V_{DD}$ droop event. During droop recovery, they low-pass filter $V_{DD}$ transients and overshoots as the voltage ultimately settles to its nominal value.

\begin{figure}[h]
	\centering
	\includegraphics[width=\columnwidth]{fig_detail_pll}
	\caption{Frequency control logic for PLL-based adaptive clocking in \cite{hashimoto2018}.}
	\label{fig:detail_pll}
\end{figure}

\begin{figure}[h]
	\centering
	\includegraphics[width=\columnwidth]{fig_detail_pllarch}
	\caption{Intra-PLL control for PLL-based adaptive clocking in \cite{hashimoto2018}.}
	\label{fig:detail_pllarch}
\end{figure}

As fast response to high-frequency supply droops is an important criterion in \cite{hashimoto2018}, significant design efforts were made to ensure low-latency clock period actuation. The reported design target is 8$\times$ faster than the first-droop frequency of 50MHz. This permits a time window of 2.5ns from the time $V_{DD}$ crosses the first supply sensor threshold to the time the SoC clock frequency is corrected. This narrow timing constraint is met by generating and propagating the critical oscillator control signal $\Delta F_{code}$ asynchronously. Crucially, the 8-bit thermometer-coded $\Delta F_{code}$ controls a DCO that can tolerate asynchronous changes in its code without glitching its output clock. The DCO reported in \cite{hashimoto2018} is a bank of nine ring inverters with eight of the rings independently enabled by each thermometer bit of $\Delta F_{code}$. Such a structure can tolerate asynchronous arrival of $\Delta F_{code}$ when suddenly decreasing frequency, as the case of a droop. Note that frequency increases are restricted: the combination of non-linear filter memory with the masking of minimum function ensures $\Delta F_{code}$ only increases one bit at a time after droop recovery \cite{hashimoto2018}. This obviates synchronization of $\Delta F_{code}$, reducing adaptation latency.

The reduction in adaptation latency directly correlates with reported experimental results. Fast response to droops allows for tighter guardbanding and therefore higher $f_{MAX}$. While \cite{hashimoto2018} does not report the exact clock adaptation latency, the work improves $f_{MAX}$ by 7.5\%, from 4.65GHz without adaptive clocking to 5.00GHz with the proposed scheme. This is the largest frequency gain reported in compared works.

\subsection{ACD-Based Adaptive Clocking}
\label{sec:details_acd}
The scheme reported in \cite{wilcox2015} presents a ACD-based adaptive clocking control for AMD's Steamroller CPU in 28nm CMOS. The power supply sensor consists of a DLL-based droop detector acting as a delay line connected to the SoC core $V_{DD}$: as $V_{DD}$ changes, the DLL phases change with respect to a regulated PLL's output clock. A programmable threshold selects one of four DLL phases to compare against the reference PLL phase to signal a 1-bit droop activity.

Fig.\ \ref{fig:detail_acd} illustrates the reported scheme. A DLL-based phase rotator realizes the ACD circuit. When a droop activity is detected and $DroopDetected$ is asserted, phase rotation between the 40 DLL phases occurs to effectively ``stretch" the clock period and reduce the clock frequency. To ensure glitchless phase rotation, small positive pre-programmed phase increments occur at each step, and the phase rotator rotates through all 40 phases before continuing the cycle. Rotation can be bypassed entirely by de-asserting $DroopEnable$. 


\begin{figure}[h]
	\centering
	\includegraphics[width=0.7\columnwidth]{fig_detail_acd}
	\caption{Frequency control logic for ACD-based adaptive clocking in \cite{wilcox2015}.}
	\label{fig:detail_acd}
\end{figure}

This scheme reports to target fast initial droops of 100MHz and slower 1-5MHz secondary droop transients by means of the programmed droop detector threshold \cite{wilcox2015}. As the detector signal is binary, the ACD-based reduces the SoC clock frequency by a fixed amount (phase rotator steps through a fixed phase step per cycle) or does not change the frequency at all. Voltage recovery is triggered when $V_{DD}$ crosses back beyond the detector threshold. The work reports a total adaptation latency of 3 cycles at 3.4GHz from droop detection to clock frequency reduction. While experimental results do not report frequency gain as in \cite{hashimoto2018}, silicon results demonstrated a reduction of 3-6\% in $V_{MIN}$, the minimum supply voltage tolerable at a given frequency for the SoC. No comparison was provided with prior works.

\section{Comparison of ACD and PLL Actuators}
\label{sec:comparison}

\subsection{The Role of Power-Supply Sensors}

The latency and accuracy of power-supply sensing is a vital facet of adaptive clock generation. Sensors must generally operate at frequencies far greater than the maximum expected supply-droop frequency, typically closer to the digital clock frequency. Both voltage-based and time-based sensors have been reported in high-performance adaptive clocking systems. These sensors form a low-resolution supply-measurement ADC, often with non-linear thresholds dictated by logic timing thresholds. Sensor designs include trade-offs between accuracy, speed of response, and digital integrate-ability, i.e. the amount of highly custom and analog circuitry which much be designed and integrated in an otherwise digital system. 

A distinct class of adaptive-clock systems omit a power-supply sensor altogether, and instead use system-level behavior information to provide \texit{feedforward droop notification}. A system-management hardware block notifies each clock-actuator shortly before major system actions which are expected to cause droops, e.g. the enabling of a large or power-hungry sub-component. 

This work compares ACD and PLL-based \textit{actuators}, independent of power-supply sensor. Designs of either popular class of sensor (voltage and time-based) are compatible with both classes of actuator examined here. The comparisons in this work proceed under the realizations that ACD and PLL systems have no inherent advantages with regards to:

\begin{enumerate}[(a)]
\item Latency in response to similar sensors, or
\item Compatibility with similar diversity of sensors, or
\item Requirements for the performance of their sensors
\end{enumerate}

In short, any sensor which works for an ACP should work equally well for an adaptive PLL. For sake of our comparisons, we assume a single-threshold supply-sensor with between one and a few cycles of latency. Such performance would be typical of analog comparator-based sensors or of feedforward droop notification. 

\subsection{Comparison Criteria}

The most obvious downside to the ACD-based systems is the inclusion of the ACD circuit itself. While low-cost variants using simple frequency-dividers have been proposed, these systems generally lack the frequency resolution to prevent significant performance degradations from minor supply disturbances. Higher resolution ACD systems such as \cite{wilcox2015} instead use a combination of delay-locked loops (DLLs) with phase rotation and/or interpolation to extend the clock period. These systems can, in principle, have period resolution of a single delay-stage delay (or less). Their primary cost is power. The DLL, phase rotation, and its accompanying control logic all operate at the logic-clock frequency. Worse, these circuits must be active at essentially all times; disabling them greatly expands the latency of the period-actuator. 

Adaptive PLL systems offer a means to remove the high-power ACD circuit altogether, by feeding their supply-sensor data near-directly into their oscillator frequency-control. These power savings may often be on the order of magnitude of the PLL itself. Digital PLLs employing ring oscillators have essentially unlimited bandwidth from frequency-control-code changes to their output frequency, allowing for near-instantaneous frequency changes upon droop detection. PLLs such as \cite{hashimoto2018} have reported actuator response times of less than a single oscillator clock cycle. A digital PLL's highly configurable control loop also provides ample mechanisms to temporarily modify its loop dynamics, and to apply frequency-recovery schemes of near-arbitrary shape and duration. 

Many of the costs of adaptive PLL systems are instead borne at design-time. Incorporating the power-supply sensor feedback inside of the PLL feedback loop complicates its dynamics and overall design. Oscillator requirements for example those for frequency range and resolution, which typically dictate PLL power through their impact on noise and stability, are further complicated by the requirement for fast frequency shifting. 

\subsection{Comparison Methodology}

The relative advantages of ACD and PLL systems manifest at different times. ACD-based systems reduce design-time complexity in exchange for run-time power and area. Adaptive PLLs, in contrast, can principally be designed at lower power, but with a more constrained and sensitive design process. Our comparison then proceeds along two lines:

\begin{enumerate}[(a)]
\item First, we examine the power and area requirements for an ACD-based actuator similar to that of \cite{wilcox2015}.
\item Second, we present the design of an adaptive PLL similar to \cite{hashimoto2018}, which we expect will alleviate many of the prominent design-time constraints. 
\end{enumerate}

Our results then compare both the realized performance, power, and area of this PLL, along with its design process, flexibility, and portability to new process technologies. 

\section{Conclusion}
\label{sec:conclusion}

This work compares the costs and effectiveness of ACD and PLL based adaptive clocking systems. ACD-based systems reduce design-time complexity in exchange for run-time power and area. We first examine the required power and area of state of the art ACD actuators. We then present the design of a digital adaptive PLL which removes the need for an ACD-based actuator, along with many of the undesirable trade-offs common in adaptive PLL design.


\bibliographystyle{IEEEtran}
\begingroup
\raggedright
\bibliography{references}
\endgroup

\end{document}
