%% DO NOT CHANGE
\documentclass{IEEEtran}
\usepackage{cite}
\usepackage{amsmath,amssymb,amsfonts}
\usepackage{algorithmic}
\usepackage{graphicx}
\usepackage{textcomp}
\def\BibTeX{{\rm B\kern-.05em{\sc i\kern-.025em b}\kern-.08em
    T\kern-.1667em\lower.7ex\hbox{E}\kern-.125emX}}
%% DO NOT CHANGE

% Run through MikTeX TexWorks once with auto-install missing packages enabled.

\begin{document}
\title{  Adaptive Clock Generation PLL and Clock Distribution }

\author{
	Dan Fritchman, \IEEEmembership{Member, IEEE} and Wahid Rahman \IEEEmembership{Member, IEEE}
	\thanks{Date of publication Mar. 20, 2020.}
	\thanks{
		D. Fritchman and W. Rahman are with the Department of Electrical Engineering and Computer Sciences, University of California, Berkeley, Berkeley, CA 94720 USA (e-mail: dan\_fritchman@berkeley.edu; wahid.rahman@berkeley.edu).}
}

\maketitle

\begin{abstract}

(This is worded more like an intro, probably make it more concise for the abstract.)

Adaptive clock generation techniques have emerged in recent generations of high-performance SoCs for mitigation of timing failures due to transient supply voltage droops. Rather than design-in timing margin via either (a) increased supply voltage, (b) improved supply-distribution, or (c) reduced supply noise, adaptive-clock systems instead detect transient supply events and temporarily reduce their clock frequency. Works such as [4], [5], and [6] include a discrete adaptive clock distribution (ACD) circuit, inserted in-line after a typical PLL. Other works such as [1], [2], and [3] instead utilize adaptive PLLs, which directly update their oscillator frequency following supply-droop events. While adaptive clocking has been shown possible for analog PLLs in [3], its cost of implementation is far lower for digital PLLs such as [1] and [2]. Such all-digital PLLs are further desirable for their small area, portability, and streamlined circuit design process. This work compares ACD and PLL-based means of adapting digital clock frequency to transient supply-voltage droops, using [1] and [4] as seminal examples of each. Finally we present the design of an adaptive all-digital PLL with sub-cycle reaction time which avoids many of the trade-offs cited in prior works [4], [5], and [6].
\end{abstract}

\begin{IEEEkeywords}
Adaptive clocking, adaptive frequency, power efficiency, supply-droop mitigation, supply-voltage droop.
\end{IEEEkeywords}

\section{Introduction}

Adaptive clock generation techniques have emerged in recent generations of high-performance SoCs for mitigation of timing failures due to transient supply voltage droops. Rather than design-in timing margin via either (a) increased supply voltage, (b) improved supply-distribution, or (c) reduced supply noise, adaptive-clock systems instead detect transient supply events and temporarily reduce their clock frequency. Works such as [4], [5], and [6] include a discrete adaptive clock distribution (ACD) circuit, inserted in-line after a typical PLL. Other works such as [1], [2], and [3] instead utilize adaptive PLLs, which directly update their oscillator frequency following supply-droop events. While adaptive clocking has been shown possible for analog PLLs in [3], its cost of implementation is far lower for digital PLLs such as [1] and [2]. Such all-digital PLLs are further desirable for their small area, portability, and streamlined circuit design process. This work compares ACD and PLL-based means of adapting digital clock frequency to transient supply-voltage droops, using [1] and [4] as seminal examples of each. Finally we present the design of an adaptive all-digital PLL with sub-cycle reaction time which avoids many of the trade-offs cited in prior works [4], [5], and [6].

\section{Adaptive Clocking Schemes}

@dan to complete this section

All adaptive clocking schemes include two fundamental components:

* (a) A *power-supply sensor*, which measures and reports transient droops in supply voltage, and
* (b) A *clock period actuator*, which modulates the system clock period in response to reports from the power-supply sensor

Additional logic or circuitry then manages recovery from these supply-induced slow-downs. 

Adaptive Clock Distribution (ACD) and PLL-based systems differ in their implementation of component (b), the clock-period actuator. ACD-based systems extend their clock period via a discrete ACD circuit, placed in-line between the PLL output clock pclk and the global logic clock gclk, as in Figure X below.

(acd.png)

Adaptive PLL systems, in contrast, directly incorporate supply-droop information into their oscillator frequency, as in Figure Y. 

(pll.png)

(A note on how we're not evaluating or designing the *sensor*, just the *actuator*.)

Power-supply sensors may be asynchronous or clocked. Generally they must operate at frequencies far greater than the maximum expected supply-droop frequency, typically closer to the digital clock frequency. Both voltage-based and time-based sensors have been reported in high-performance adaptive clocking systems. These sensors form a low-resolution supply-measurement ADC, often with non-linear thresholds dictated by logic timing thresholds. Sensor designs include trade-offs between accuracy, speed of response, and digital integrate-ability, i.e. the amount of highly custom and analog circuitry which much be designed and integrated in an otherwise digital system. Designs of either popular class of sensor (voltage-based and time-based) are compatible with both classes of actuator examined here. Design of these sensors is not explored in this work. We assess and design clock-actuators expecting a supply-sensor supporting (some feature list of thresholds, latency, etc.). 


\subsection{Post-PLL modulation}

@wahid
Discuss \cite{wilcox2015}.

\subsection{Direct modulation of PLL}

@wahid
Discuss \cite{hashimoto2018}.

PLL characteristics that help this:

* Instant frequency change
* Digitally manageable recovery
* Knowledge of control versus frequency info, i.e. calibration data 
* Instantly reconfigurable loop filter

\section{Comparison of ACD and PLL Actuators}

@dan
Key performance metrics and measurement setup for comparison between \cite{hashimoto2018} and \cite{wilcox2015}.

\section{Comparison}
Direct comparison of \cite{hashimoto2018} and \cite{wilcox2015}. Clear hypothesis of which design is more viable.

\section{Conclusion}

Our "very clear hypothesis" is that we can make a PLL, that doesn't suck. 
In continued work, we will further investigate ... doing that.

@dan to complete this section

\bibliographystyle{IEEEtran}
\begingroup
\raggedright
\bibliography{references}
\endgroup

\end{document}
