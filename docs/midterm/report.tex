%% DO NOT CHANGE
\documentclass{IEEEtran}
\usepackage{cite}
\usepackage{amsmath,amssymb,amsfonts}
\usepackage{algorithmic}
\usepackage{graphicx}
\usepackage{textcomp}
\def\BibTeX{{\rm B\kern-.05em{\sc i\kern-.025em b}\kern-.08em
    T\kern-.1667em\lower.7ex\hbox{E}\kern-.125emX}}
%% DO NOT CHANGE

% Run through MikTeX TexWorks once with auto-install missing packages enabled.

\begin{document}
\title{  Adaptive Clock Generation PLL and Clock Distribution }

\author{
	Dan Fritchman, \IEEEmembership{Member, IEEE} and Wahid Rahman \IEEEmembership{Member, IEEE}
	\thanks{Date of publication Mar. 20, 2020.}
	\thanks{
		D. Fritchman and W. Rahman are with the Department of Electrical Engineering and Computer Sciences, University of California, Berkeley, Berkeley, CA 94720 USA (e-mail: dan\_fritchman@berkeley.edu; wahid.rahman@berkeley.edu).}
}

\maketitle

\begin{abstract}
Adaptive clock generation techniques have emerged in recent generations of high-performance SoCs for mitigation of timing failures due to transient supply voltage droops. Rather than design-in timing margin via either (a) increased supply voltage, (b) improved supply-distribution, or (c) reduced supply noise, adaptive-clock systems instead detect transient supply events and temporarily reduce their clock frequency. Works such as [4], [5], and [6] include a discrete adaptive clock distribution (ACD) circuit, inserted in-line after a typical PLL. Other works such as [1], [2], and [3] instead utilize adaptive PLLs, which directly update their oscillator frequency following supply-droop events. While adaptive clocking has been shown possible for analog PLLs in [3], its cost of implementation is far lower for digital PLLs such as [1] and [2]. Such all-digital PLLs are further desirable for their small area, portability, and streamlined circuit design process. This work compares ACD and PLL-based means of adapting digital clock frequency to transient supply-voltage droops, using [1] and [4] as seminal examples of each. Finally we present the design of an adaptive all-digital PLL with sub-cycle reaction time which avoids many of the trade-offs cited in prior works [4], [5], and [6].
\end{abstract}

\begin{IEEEkeywords}
Adaptive clocking, adaptive frequency, power efficiency, supply-droop mitigation, supply-voltage droop.
\end{IEEEkeywords}

\section{Introduction}
Summarize problem, state of the art

\section{Background}
\subsection{Direct modulation of PLL}
Discuss \cite{hashimoto2018}.
\subsection{Post-PLL modulation}
Discuss \cite{wilcox2015}.

\section{Comparison setup}
Key performance metrics and measurement setup for comparison between \cite{hashimoto2018} and \cite{wilcox2015}.

\section{Comparison}
Direct comparison of \cite{hashimoto2018} and \cite{wilcox2015}. Clear hypothesis of which design is more viable.

\section{Conclusion}
In continued work, we will further investigate ...


\bibliographystyle{IEEEtran}
\begingroup
\raggedright
\bibliography{references}
\endgroup

\end{document}
